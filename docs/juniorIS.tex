\documentclass[12pt]{article}

\title{Virtual Synthesizer Development in C++}
\date{\vspace{-5ex}}
\author{Will Sieber}

\usepackage[left=1in,right=1in,top=1in,bottom=1in]{geometry}
\usepackage{amsmath}
\usepackage{hyperref}
\usepackage{setspace}

\spacing{1.5}

\begin{document}

\maketitle

% Latex can automatically build a table of contents for you
\tableofcontents
\newpage


\section{Introduction}

\subsection{Project Goals}


\section{Synthesizer Basics}

\subsection{Components}

\subsubsection{Oscillators}
Oscillators can be generally broken into two categories: tone generators and controllers. Tone generators repeat periodic waves in order to generate sound. Simple tone generators can be expressed as mathematical expressions, such as \(f(x) = sin(x)\), while complex tone generators allow for any waveform to be repeated, regardless if they can be expressed as a single expression. Complex oscillators generally allow for a number of different waveforms to be represented in a single table, allowing for a user to select any number of waveforms in real time. Controller oscillators, on the other hand, are much lower frequency than tone oscillators, thus have the name \textit{Low-Frequency Oscillators} (LFOs). LFOs are generally used as an input to automate other parameters of a synthesizer, allowing for a greater range of timbres. For the purposes of this project, simple oscillators are used as tone generators. 

\subsubsection{Envelope Generators}
In order to provide articulation, envelope generators describe how the amplitude of a particular sound should change over time. Several parameters are used to accomplish this goal, and those are the synthesizer's \textit{attack}, \textit{decay}, \textit{sustain}, and \textit{release}. \textit{Attack} describes how much time it takes for the synthesizer to reach full amplitude upon receiving the signal to generate sound. A short attack will reach full amplitude quickly, such as pressing a key on a piano or organ. Longer attacks will take more time to reach full amplitude, which create a "swelling" effect that is similar to that of a violin increasing in volume. \textit{Decay} describes how much time the synthesizer should take to reach a particular amplitude once the attack is finished. A longer decay will prolong a sound prior to the sustain, while a shorter will reach it quicker [word better]. \textit{Sustain}, unlike attack and decay, describes the amplitude itself to which the decay will reach upon completion. When reached, the synthesizer will continuously play at this amplitude until a signal is sent that it should stop generating sound. Upon this signal, \textit{release} describes how long the synthesizer should take to stop generating sound by gradually decreasing the amplitude to zero. In more technical terms []. 

One may notice a problem with this model - the rate at which the attack, decay, and release change amplitude itself is not described. Powerful synthesizers allow for this change to be represented as a polynomial function, but for the purposes of this project, the rate at which each parameter changes will simply be linear. 

With these two components in mind, one can create a simple synthesizer in the same manner that is described in this project.

\subsection{Basic Types of Synthesis}

\subsubsection{Additive Synthesis}

\subsubsection{Subtractive Synthesis}

\subsubsection{Wavetable Synthesis}

\subsubsection{FM Synthesis}

\subsection{Architecture and Prototyping}

\subsubsection{Pure Data / Max MSP}

\subsubsection{Architecture Diagram}

\section{Implementation}

\subsection{Theory}

\subsubsection{Discrete Signals}

\subsubsection{Buffers}

\subsubsection{Flow of Information}

\subsection{Comparison of Frameworks}

\subsection{Introduction to the VST3 SDK}

\subsubsection{VST GUI and Development Tools}

\subsubsection{Workflow and UML diagram}

\subsection{Creating Oscillators}

\subsection{Applying ADSR}

\section{Effects and Extras}

\section{Conclusion}


% Bibliography
\bibliography{bibliography}
\bibliographystyle{abbrv}

\end{document}