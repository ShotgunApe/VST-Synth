\documentclass[12pt]{article}

\title{Virtual Synthesizer Development in C++}
\date{\vspace{-5ex}}
\author{Will Sieber}

\usepackage[left=1in,right=1in,top=1in,bottom=1in]{geometry}
\usepackage{amsmath}
\usepackage{hyperref}
\usepackage{setspace}

\spacing{1.5}

\begin{document}

\maketitle

% Latex can automatically build a table of contents for you
\tableofcontents
\newpage


\section{Introduction}

\subsection{Project Goals}

\subsection{Glossary of Terms}

\section{Synthesizer Basics}

\subsection{Components}

\subsubsection{Oscillators}

\subsubsection{Envelope Generators}

\subsubsection{Filters}

\subsubsection{Digitally Controlled Amplifiers}

\subsubsection{Low-Frequency Oscillators}

\subsection{Basic Types of Synthesis}

\subsubsection{Subtractive Synthesis}

\subsubsection{Additive Synthesis}

\subsubsection{Wavetable Synthesis}

\subsubsection{FM Synthesis}

\subsection{Architecture and Prototyping}

\subsubsection{Pure Data / Max MSP}

\subsubsection{Architecture Diagram}

\section{Implementation}

\subsection{Comparison of Frameworks}

\subsection{Introduction to the VST3 SDK}

\subsubsection{Installation}

\subsubsection{VST GUI and Development Tools}

\subsection{Creating Oscillators}

\subsection{Applying ADSR}

\section{Effects and Extras}

\section{Conclusion}


% Bibliography
\bibliography{bibliography}
\bibliographystyle{abbrv}

\end{document}